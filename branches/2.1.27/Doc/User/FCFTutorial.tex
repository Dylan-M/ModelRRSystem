%* 
%* ------------------------------------------------------------------
%* FCFTutorial.tex - Tutorial chapter for the Freight Car Forwarder (V2)
%* Created by Robert Heller on Sat Apr 21 11:17:20 2007
%* ------------------------------------------------------------------
%* Modification History: $Log$
%* Modification History: Revision 1.2  2007/09/30 15:48:32  heller
%* Modification History: Rev 2.1.10 Lockdown
%* Modification History:
%* Modification History: Revision 1.1  2007/05/06 12:49:40  heller
%* Modification History: Lock down  for 2.1.8 release candidate 1
%* Modification History:
%* Modification History: Revision 1.1  2002/07/28 14:03:50  heller
%* Modification History: Add it copyright notice headers
%* Modification History:
%* ------------------------------------------------------------------
%* Contents:
%* ------------------------------------------------------------------
%*  
%*     Model RR System, Version 2
%*     Copyright (C) 1994,1995,2002-2005  Robert Heller D/B/A Deepwoods Software
%* 			51 Locke Hill Road
%* 			Wendell, MA 01379-9728
%* 
%*     This program is free software; you can redistribute it and/or modify
%*     it under the terms of the GNU General Public License as published by
%*     the Free Software Foundation; either version 2 of the License, or
%*     (at your option) any later version.
%* 
%*     This program is distributed in the hope that it will be useful,
%*     but WITHOUT ANY WARRANTY; without even the implied warranty of
%*     MERCHANTABILITY or FITNESS FOR A PARTICULAR PURPOSE.  See the
%*     GNU General Public License for more details.
%* 
%*     You should have received a copy of the GNU General Public License
%*     along with this program; if not, write to the Free Software
%*     Foundation, Inc., 675 Mass Ave, Cambridge, MA 02139, USA.
%* 
%*  
%* 

\chapter{Freight Car Forwarder (V2) Tutorial}
\label{chpt:fcf:Tutorial}
\typeout{$Id$}


\index{Freight Car Forwarder!Tutorial|(} The Freight Car Forwarder is a
program designed to simulate freight car traffic on your model
railroad.  It does this by matching types of freight cars with
industries.  Specific types of freight cars are meant to carry specific
types of commodities and specific industries produce or consume
specific types of commodities.

Before you start using the Freight Car Forwarder system, you should
carefully study Section~\ref{sect:fcf:Files} and 
Section~\ref{sect:fcf:File Formats} of the reference section.   These
files describe the system layout (system file), the industries
(industry file), the trains that will move the cars (trains file), and
the cars themselves (the cars file).  There are some additional files,
including an owner's file and a car types file, as well as a file for
statistics.  All of these files are plain text files--you will need a
plain text file editor (such as Notepad under MS-Windows or gedit under
many versions of Linux\footnote{It might be worthwhile to install a
powerful general purpose text editor such as GNUEmacs for this
puspose.}) to create and generally edit these files. The only files
that are ever modified by the Freight Car Forwarder system are the cars
files and the statistics file. You should \emph{not} edit the statistics
file--this file is automatically generated by the Freight Car Forwarder
system.  While it is possible to use tools available as part of the
Freight Car Forwarder system to edit cars in the cars file, it is
probably best to use a regular text editor to add, modify, or delete
cars in a wholesale manor.  All of the other files are treated as
``constant data'' by the Freight Car Forwarder system, which will load
the data into memory and not modify that data.

\section{Loading System Data}

The Freight Car Forwarder starts loading data by opening and reading
the system file, using either the file menu's \verb=Open...= item or open file
button on the toolbar.  This file contains the path names of the other
files, which are assumed to be relative to the directory (folder) that
contains the system file.  All of the system data is loaded into a
large data structure, which is then used by the program to simulate car
movements.

\section{Assigning Cars}

  In order to move cars, the cars need to be \emph{assigned}, that is, they
need to have a destination set, either to be loaded (if empty) or
unloaded (if loaded).  The Car Assignment procedure performs this task.

\section{Running Trains}

Once cars has been assigned, they need to be moved.  Cars are moved on
trains, and this is done with the run trains procedures.  There are
three of these procedures: Run All Trains in Operating Session, Run
Boxmoves, and Run One Train at a time.  The run trains procedures
simulate the actual movement of cars and determines which trains will
move which cars and in what order.  From this simulation, a set of yard
and switch lists can be generated and printed out for use during your
operating session.

\section{Printing Yard and Switch Lists}

Once the trains have been run, yard and switch lists can be printed
out, using the print yard lists menu.

\section{Saving the updated car data}

Once you have assigned cars, simulated train movements and created your
Yard and Switch Lists, you should save the cars file.  You should now
be ready to run your trains in an operating session.  If your session
went off well, you will be ready to make car assignments for your next
session.  If there were problems during the operating session (such as
bad ordered cars or late trains), you might have to make adjustments
before running the car assignment process for the next session.  You
would use the car editor to make these adjustments\footnote{It might be
easier to use a text editor in the case of wholesale changes.}.

\section{Generating Reports}

Various reports can also be generated and printed using the reports
menu. 

\section{Other activities}

Other activities include adding, removing, and editing cars and
displaying various state information, such as assigned and unassigned
cars, car movement information, lists of trains, and lists of
industries, stations, and divisions.
\index{Freight Car Forwarder!Tutorial|)} 

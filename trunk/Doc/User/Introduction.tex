%* 
%* ------------------------------------------------------------------
%* Introduction.tex - Introduction to the user manaual
%* Created by Robert Heller on Sat Apr 21 10:33:48 2007
%* ------------------------------------------------------------------
%* Modification History: $Log$
%* Modification History: Revision 1.1  2007/05/06 12:49:40  heller
%* Modification History: Lock down  for 2.1.8 release candidate 1
%* Modification History:
%* Modification History: Revision 1.1  2002/07/28 14:03:50  heller
%* Modification History: Add it copyright notice headers
%* Modification History:
%* ------------------------------------------------------------------
%* Contents:
%* ------------------------------------------------------------------
%*  
%*     Model RR System, Version 2
%*     Copyright (C) 1994,1995,2002-2005  Robert Heller D/B/A Deepwoods Software
%* 			51 Locke Hill Road
%* 			Wendell, MA 01379-9728
%* 
%*     This program is free software; you can redistribute it and/or modify
%*     it under the terms of the GNU General Public License as published by
%*     the Free Software Foundation; either version 2 of the License, or
%*     (at your option) any later version.
%* 
%*     This program is distributed in the hope that it will be useful,
%*     but WITHOUT ANY WARRANTY; without even the implied warranty of
%*     MERCHANTABILITY or FITNESS FOR A PARTICULAR PURPOSE.  See the
%*     GNU General Public License for more details.
%* 
%*     You should have received a copy of the GNU General Public License
%*     along with this program; if not, write to the Free Software
%*     Foundation, Inc., 675 Mass Ave, Cambridge, MA 02139, USA.
%* 
%*  
%* 

\chapter{Introduction}
\label{chpt:Introduction}
\typeout{$Id$}

\section{How this manual is organized.}

This manual is broken up into ``parts'', one for each ``main program''%
\footnote{The various programming libraries are described in the programming
guides\cite{progguide}.} Each part contains two or more chapters, a
tutorial chapter and one or more reference chapters.  These parts are:

\begin{itemize}
\item Part~\ref{part:univtest} documents the Universal Test program. This
program is part of the CMR/I (Chubb) library and is used to test CMR/I
nodes.
\item Part~\ref{part:timetable} documents the Time Table (V2) program.
This program is used to create employee timetables.
\item Part~\ref{part:fcf} documents the Freight Car Forwarder (V2)
program. This program is used to create switch lists for freight car
forwarding.
\item Part~\ref{part:calc} documents the calculator
scripts\footnote{Presently only Resistor.} that are available to help
model railroaders perform some common calculates.
\item Part~\ref{part:camera} documents the camera scripts.  These
scripts perform various camera scene calculations that are useful for
model railroaders.
\item Part~\ref{part:dispatch} documents the automated dispatcher program.
%\item Part~\ref{part:mainprogram} documents the generic main program.
\end{itemize}

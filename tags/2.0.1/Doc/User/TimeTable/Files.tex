%* 
%* ------------------------------------------------------------------
%* Model Railroad System by Deepwoods Software
%* ------------------------------------------------------------------
%* Files.tex - Data file formats
%* Created by Robert Heller on Sat Mar  9 09:34:43 2002
%* ------------------------------------------------------------------
%* Modification History: $Log$
%* Modification History: Revision 1.1  2002/11/09 21:21:07  heller
%* Modification History: Time Table User Manual
%* Modification History:
%* ------------------------------------------------------------------
%* Contents:
%* ------------------------------------------------------------------
%*  
%*     Model RR System, Version 2
%*     Copyright (C) 1994-2002  Robert Heller D/B/A Deepwoods Software
%* 			51 Locke Hill Road
%* 			Wendell, MA 01379-9728
%* 
%*     This program is free software; you can redistribute it and/or modify
%*     it under the terms of the GNU General Public License as published by
%*     the Free Software Foundation; either version 2 of the License, or
%*     (at your option) any later version.
%* 
%*     This program is distributed in the hope that it will be useful,
%*     but WITHOUT ANY WARRANTY; without even the implied warranty of
%*     MERCHANTABILITY or FITNESS FOR A PARTICULAR PURPOSE.  See the
%*     GNU General Public License for more details.
%* 
%*     You should have received a copy of the GNU General Public License
%*     along with this program; if not, write to the Free Software
%*     Foundation, Inc., 675 Mass Ave, Cambridge, MA 02139, USA.
%* 
%*  
%* 

\chapter{Data File Formats}
\label{chapt:Files}

All data files used by this program are plain text files.  The can be
edited with a plain text editor, although this is not recommended in
general.

\section{Cabs File}

Cab files have an extension of \verb=.cabs=, and are formatted one cab
per line as shown in Figure~\ref{fig:cabfile}.

\begin{figure}
\begin{centering}
\verb=cab name|cab color=\\
\caption{Cab File line syntax}
\label{fig:cabfile}
\end{centering}
\end{figure}

\section{Stations File}

Station files an extension of \verb=.stations=, and are formatted with
the first line containing the total end-to-end run length in $smiles$
(or $skilometers$), followed by each station as shown in
Figure~\ref{fig:stationFileStations}, then comes the duplicate trackage
map as shown in Figure~\ref{fig:stationFileDupTracks}, which is flagged
by a line consisting of \verb=%%% Duplicate Trackage Map=.

\begin{figure}   
\begin{centering}
\verb=station|distance=
\caption{Station line in the Stations file}
\label{fig:stationFileStations}
\end{centering}
\end{figure}

\begin{figure}   
\begin{centering}
\verb"station=station|station=station"
\caption{Duplicate Trackage line in the Stations file}
\label{fig:stationFileDupTracks}
\end{centering}
\end{figure}


\section{Storage Track List File}

Storage Track files have an extension of \verb=.tracks=, and are
formatted one station per line as shown in Figure~\ref{fig:trackfile}.

\begin{figure}
\begin{centering}
\verb=station|track list=\\
\caption{Storage Track File  line syntax}
\label{fig:trackfile}
\end{centering}
\end{figure}


\section{Chart File}

Chart files have an extension of \verb=.chart=, and are formatted as a
series of sections, set off by lines beginning with three percent signs
(\verb=%%%=).  The first section is the time scale section, formatted
as shown in Figure~\ref{fig:chtimescale}.  Then the cab color section,
headed with the line, \verb=%%%CABCOLORS:= and then formatted as shown
in Figure~\ref{fig:cabfile}, then the station list, headed with a line
containing the total length, formatted as
\verb=%%%STATIONTOTALLENGTH:length=, followed by lines formatted as in
Figure~\ref{fig:stationFileStations}.  This is followed by the
duplicate track map, set off by the line \verb=%%%DUPLICATETRACKMAP:=
with lines formatted as in Figure~\ref{fig:stationFileDupTracks}.  Then
comes the storage tracks, headed by the line \verb=%%%STORAGETRACKS:=
with lines formatted as in Figure~\ref{fig:trackfile}.  After this
comes the trains, which are set off with the line \verb=%%%TRAINS:= and
formatted as in Figure~\ref{fig:chfileTrains}.  Then comes the storage
track usage map, headed by the line \verb=%%%STORAGETRACKMAP:= with
lines formatted as shown in Figure~\ref{fig:chfileStorUseMap}. See the
Internals Manual for details on the format of the train array and the
storage track usage map array.  Finally the notes are stored after a
heading line of \verb=%%%NOTES:= as lines formatted as in
Figure~\ref{fig:chfileNotes}. The note entries have had their internal
new line characters replaced by the two character sequence \verb=\n=.

\begin{figure}
\begin{centering}
\verb=%%%TIMESCALE: TotalTime TimeIncrement=\\
\caption{Time scale line in Chart file}
\label{fig:chtimescale}
\end{centering}
\end{figure}

\begin{figure}
\begin{centering}
\verb=train|train structure=\\
\caption{Train line in chart file}
\label{fig:chfileTrains}
\end{centering}
\end{figure}

\begin{figure}
\begin{centering}
\verb=storage key|storage usage=\\
\caption{Storage Track Usage line in chart file}
\label{fig:chfileStorUseMap}
\end{centering}
\end{figure}

\begin{figure}
\begin{centering}
\verb=note number|note text=\\
\caption{Note line in chart file}
\label{fig:chfileNotes}
\end{centering}
\end{figure}


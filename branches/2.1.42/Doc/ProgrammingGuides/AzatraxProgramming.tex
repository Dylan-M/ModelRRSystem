%* 
%* ------------------------------------------------------------------
%* AzatraxProgramming.tex - Using Azatrax device interfaces
%* Created by Robert Heller on Fri Dec  2 09:35:04 2011
%* ------------------------------------------------------------------
%* Modification History: $Log$
%* Modification History: Revision 1.1  2002/07/28 14:03:50  heller
%* Modification History: Add it copyright notice headers
%* Modification History:
%* ------------------------------------------------------------------
%* Contents:
%* ------------------------------------------------------------------
%*  
%*     Model RR System, Version 2
%*     Copyright (C) 1994,1995,2002-2005  Robert Heller D/B/A Deepwoods Software
%* 			51 Locke Hill Road
%* 			Wendell, MA 01379-9728
%* 
%*     This program is free software; you can redistribute it and/or modify
%*     it under the terms of the GNU General Public License as published by
%*     the Free Software Foundation; either version 2 of the License, or
%*     (at your option) any later version.
%* 
%*     This program is distributed in the hope that it will be useful,
%*     but WITHOUT ANY WARRANTY; without even the implied warranty of
%*     MERCHANTABILITY or FITNESS FOR A PARTICULAR PURPOSE.  See the
%*     GNU General Public License for more details.
%* 
%*     You should have received a copy of the GNU General Public License
%*     along with this program; if not, write to the Free Software
%*     Foundation, Inc., 675 Mass Ave, Cambridge, MA 02139, USA.
%* 
%*  
%* 


\chapter{Using the Azatrax device Interfaces.}
\label{chapt:Azatrax:AzatraxProgramming}
\typeout{$Id$}

\section{aztrax::Azatrax Class}
\label{sect:Azatrax:AzatraxClass}

\begin{lstlisting}[caption={azatrax::Azatrax Class (simplified public interface)}, 
		  language=C++,label=lst:Azatrax:AzatraxClass]
class aztrax::Azatrax {
public:
	~Azatrax();
	static int NumberOfOpenDevices();
	static char ** AllConnectedDevices();
	static Azatrax *OpenDevice(const char *serialnumber, 
				unsigned short int idProduct=0,
				char **outmessage=NULL);
	friend class aztrax::MRD;
	friend class aztrax::SL2;
	friend class aztrax::SR4;
	enum {
		idAzatraxVendor,
		idMRDProduct,
		idSL2Product,
		idSR4Product 
	};
	ErrorCode RestoreLEDFunction() const;
	ErrorCode Identify_1() const;
	ErrorCode GetStateData();
	uint8_t PacketCount() const;
	const char *SerialNumber() const;
	const char *MyProduct() const;
	unsigned short int MyProductId() const;
	static unsigned short int ProductIdCode(const char *productName);
};
\end{lstlisting}

The Azatrax Interface\cite{internals} has both a C++ and a Tcl API and
this chapter will cover both. Both APIs use the same underlying
code\footnote{In fact the same shared library.}. The API is based on the
C++ classes that encapsulates the various Azatrax devices, the MRD2-S or
MRD2-U (MRD class), the SR4 (the MRD class), or the SL2 (the SL2 class).
Each Azatrax device has a factory set serial number.

The Azatrax class is the base class for all of the Azatrax device
classes. Its constructor should not be called directly.  Instead the
\texttt{OpenDevice} static member should be called.  This member
function will return a device specific instance (at this time, one of
aztrax::MRD, aztrax::SR4, or aztrax::SL2), depending on the type of
device opened.  The device will be opened based on its serial number
and optionally by a specific product type.  The result of
\texttt{OpenDevice} function should be cast to a specific device class.
It will never really be a base Azatrax class instance, since the base
class has little functional usefulness.  The device specific classes
implement both the common functionallity of the base class and all of
the device specific functionallity.

\section{aztrax::MRD Class}
\begin{lstlisting}[caption={azatrax::MRD Class (simplified public interface)},
		   language=C++,label=lst:Azatrax:MRDClass]
#define ErrorCode int
class azatrax::MRD public azatrax::Azatrax{
public:
	enum OperatingMode_Type { NonTurnoutSeparate,
				  NonTurnoutDirectionSensing, 
				  TurnoutSolenoid, 
				  TurnoutMotor };
	~MRD();
	ErrorCode SetChan1() const;
	ErrorCode SetChan2() const;
	ErrorCode ClearExternallyChanged() const;
	ErrorCode DisableExternal() const;
	ErrorCode EnableExternal() const;
	ErrorCode RestoreLEDFunction() const;
	ErrorCode Identify_2() const;
	ErrorCode Identify_1_2() const;
	ErrorCode ResetStopwatch() const;
	bool Sense_1() const;
	bool Sense_2() const;
	bool Latch_1() const;
	bool Latch_2() const;
	bool HasRelays() const;
	bool ResetStatus() const;
	bool StopwatchTicking() const;
	bool ExternallyChanged() const;
	bool AllowingExternalChanges() const;
	OperatingMode_Type OperatingMode() const;
	void Stopwatch(uint8_t &fract, uint8_t &seconds, uint8_t &minutes,
			uint8_t &hours) const; }; 
\end{lstlisting} 
The MRD2-S has two relays and the MRD2-U has no relays.  Both devices
have two sensors and a stopwatch. The class instance is connected to a
device using its unique serial number, using the aztrax::Azatrax static
method, \texttt{OpenDevice}, see
Section~\ref{sect:Azatrax:AzatraxClass}, and
Listing~\ref{lst:Azatrax:AzatraxClass}. This class exposes a collection
of methods to send command bytes to the device, a method to retrieve
the device's state data and a collection of methods to access the
device's state data, once it has been retrieved.



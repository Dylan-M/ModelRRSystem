%* 
%* ------------------------------------------------------------------
%* BWHelp.tex - Using the help package.
%* Created by Robert Heller on Thu Apr 19 14:35:28 2007
%* ------------------------------------------------------------------
%* Modification History: $Log$
%* Modification History: Revision 1.3  2007/11/30 13:56:50  heller
%* Modification History: Novemeber 30, 2007 lockdown.
%* Modification History:
%* Modification History: Revision 1.2  2007/10/22 17:17:27  heller
%* Modification History: 10222007
%* Modification History:
%* Modification History: Revision 1.1  2007/05/06 12:49:38  heller
%* Modification History: Lock down  for 2.1.8 release candidate 1
%* Modification History:
%* Modification History: Revision 1.1  2002/07/28 14:03:50  heller
%* Modification History: Add it copyright notice headers
%* Modification History:
%* ------------------------------------------------------------------
%* Contents:
%* ------------------------------------------------------------------
%*  
%*     Model RR System, Version 2
%*     Copyright (C) 1994,1995,2002-2005  Robert Heller D/B/A Deepwoods Software
%* 			51 Locke Hill Road
%* 			Wendell, MA 01379-9728
%* 
%*     This program is free software; you can redistribute it and/or modify
%*     it under the terms of the GNU General Public License as published by
%*     the Free Software Foundation; either version 2 of the License, or
%*     (at your option) any later version.
%* 
%*     This program is distributed in the hope that it will be useful,
%*     but WITHOUT ANY WARRANTY; without even the implied warranty of
%*     MERCHANTABILITY or FITNESS FOR A PARTICULAR PURPOSE.  See the
%*     GNU General Public License for more details.
%* 
%*     You should have received a copy of the GNU General Public License
%*     along with this program; if not, write to the Free Software
%*     Foundation, Inc., 675 Mass Ave, Cambridge, MA 02139, USA.
%* 
%*  
%* 

\chapter{Using the Help package.}
\label{chapt:BWHelp}
\typeout{$Id$}

Included with the Model Railroad System is a simple hyperhelp package,
consisting of a Tcl package (BWHelp) and a pair of support tclkits
(HelpIndexBuild.kit and IndexHH.kit).  To include help with your Tcl
code you need to use a package require for the BWHelp package and
create a directory with the properly formatted help files, complete
with an index, created with the tclkit HelpIndexBuild.kit. 
Section~\ref{sect:BWH:BWHelpPackage} describes how to use the Tcl
package, Section~\ref{sect:BWH:HyperHelpFiles} describes how to create
help files, and Section~\ref{sect:BWH:BuildIndex} describes how to build
the indexe files.


\section{Using the BWHelp package}
\label{sect:BWH:BWHelpPackage}

The BWHelp package provides four functions and uses one global
variable.  The four functions are:

\begin{enumerate}
\item \lstinline=BWHelp::HelpTopic topic updateHistory key=\index{BWHelp!HelpTopic procedure} This
function displays the selected topic in the help window. Its arguments
are the help topic to display, a flag indicating whether or not to
update the help history, and an optional key into the help index.
\item \lstinline=BWHelp::GetTopLevelOfFocus menu=\index{BWHelp!GetTopLevelOfFocus procedure} This function returns
the toplevel that currently has the keyboard focus.  This is used when
generic pulldown menus are activated to determine which object the menu
refers to.  Its argument is the menu that was selected.  This is used to
select which display to search on.
\item \lstinline=BWHelp::HelpContext widget=\index{BWHelp!HelpContext procedure} This displays an ``On
Context'' help dialog. Its argument is the toplevel to track context on.
\item \lstinline=BWHelp::HelpWindow widget=\index{BWHelp!HelpWindow procedure} This displays help for a
selected widget, passed as its argument.
\end{enumerate}

The global variable, \lstinline=HelpDir=\index{BWHelp!HelpDir global}, is used to find the help files
containing the on-line help.  There is one special file,
\lstinline=hh.index=, that contains a mapping of topic strings to files
and offsets with the specified help topic text resides.

Normally, all you need to do is set the \lstinline=HelpDir= global
variable to the path of the directory where the help files are and bind
scripts using the \lstinline=BWHelp::HelpTopic= function to Help buttons
and Help menus\footnote{The BWStdMenuBar package includes code to create
a standard Help menu with all of the proper Help menu items. See
Chapter~\ref{chapt:MainWindows} for more information.}.

\section{Help File Formatting}
\label{sect:BWH:HyperHelpFiles}

\begin{lstlisting}[language=HTML,
		   caption={Typical help file},
		   label={lst:BWH:HelpFile}]
0 Main help
This is the main topic.  There are several sub-topics.
1 First Subtopic
This is a subtopic.
2 Subsubtopic
This is a subsubtopic.
1 Second Subtopic
This is another subtopic.
1 Third Subtopic
This is yet another subtopic.
2 A subsubtopic under the third Subtopic
Nested topic.
3 Deeper nesting
A more deeply nested topic.
\end{lstlisting}
Hyper help files are plain text files that use a combination of simple
hyper text markup and a technique ``lifted'' from VMS Help.  Each help
file contains one or more sections set off by a header line.  A header
line is a list that starts with a positive decimal number, whitespace
characters and a help topic.  The decimal number represents a help
level within a tree of help.  Level 0 is a root topic.  Higher values
represent branch and leaf topics and indicate a nesting of help topics.
Generally, each help file has at least one Level 0 or root topic, as
shown in Listing~\ref{lst:BWH:HelpFile}.

The simple markup uses pairs of delimiters: \verb=<>= for links,
\verb=[]= for bolding, and \verb={}= for inlined images.  The backslash
(\verb=\=) is used to escape these characters when they are used
literally.  Links use the topic heading as the link text.  Images can be
whatever is supported by the base Tcl/Tk code (GIF and PPM in base level
Tcl/Tk, other formats in the case the Img package is loaded).


\section{Hyper Help Build Support}
\label{sect:BWH:BuildIndex}

The BWHelp package uses a pre-computed topic indexing file, named
\verb=hh.index=, to map between help topics and help files and offsets
into help files.  The tclkit, \verb=HelpIndexBuild.kit=, is used to
generate this file.  This program takes as its first argument the name
of the indexing file and the names of the help files as the rest of its
arguments.  The help files are scanned one by one and a topic index
tree is created and stored in the indexing file, which in turn is used
by the BWHelp package to quickly find the help topic text at runtime.
This kit is documented in Section~\ref{sect:STR:helpindexbuild}.

Another script, \verb=IndexHH.kit=, can be used to create a displayable
help index (typically called index.hh), with a toplevel topic of Index.
This kit is documented in Section~\ref{sect:STR:indexhh}.


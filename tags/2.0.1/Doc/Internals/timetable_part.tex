\part{Time Table Tcl Internals}
\chapter{Introduction}
\label{chapt:Introduction}

This document describes the inner workings on the model railroad time
table generator.  The generator is encoded in three Tcl files:

\begin{enumerate}

\item mrrTimeTable.tcl.  This file contains the bulk of the code,
including the mainline code.

\item StdMenuBar.tcl.  This file contains the code that implements a
standard Motif-style menu bar.

\item MakeTimeTable.tcl.  This file contains the code to generate the
hardcopy timetable.

\end{enumerate}

The functions and global variables in each of these files is described
in the following three chapters.

%
\include{mrrTimeTable}
\include{MakeTimeTable}
\chapter{The Structure of Global Variables}
\label{chapt:GlobalStructure}
%
\section{Stations array and Duplicate Track Map}

\index{Stations!global|(}
The {\tt Stations} array contains the {\em smile}\footnote{{\em
Skilometers} will also work if train speed is entered in skph.} location
of the station, as a real number.  The array is indexed by the name of
the station.  When stations are listed, they are always listed in
distance order.

Associated with the {\tt Stations} array is the scalar variable, {\tt
TotalLength}\index{TotalLength!global}, which contains the total length
of the railroad, as a real number.  \index{Stations!global|)}

\index{DuplicateTrackMap!global|(}
The {\tt DuplicateTrackMap} array contains the mapping of track sections
that are duplicated.  These are sections of track that a train traverses
twice during a single run.  Typically these occur in in the single track
sections on out and back type layouts or dog-bone layouts that use
single track sections.  The array is indexed by a pair of stations,
separated by an equal sign ($=$) character.  The value is another pair
of stations, also separated by an equal sign character.
\index{DuplicateTrackMap!global|)}

\index{StationYMap!global|(}
The {\tt StationYMap} array contains the mapping of stations to $Y$
coordinates of the station section of the chart area.
\index{StationYMap!global|)}

\section{Total time and time increment variables}

\index{TotalTime!global|(}
\index{TimeIncrement!global|(}
The two variables, {\tt TotalTime} and {\tt TimeIncrement}, hold the
chart width and the chart line increment in fast clock minutes, as real
numbers.  The default is 24 hours for the total time and 15 minutes for
the increment.
\index{TotalTime!global|)}
\index{TimeIncrement!global|)}

\section{Cab color map}

The {\tt CabColors} array maps cab names to displayed
colors.\index{CabColors!global}


The {\tt CabYMap}\index{CabYMap!global} array contains the mapping of
cabs to $Y$ coordinates of the cab section of the chart area.

\section{Storage track variables}

The {\tt TrackList}\index{TrackList!global} array contains the list of
storage track numbers of each station that has storage tracks.

The {\tt TrackYMap}\index{TrackYMap!global} array contains the mapping
of storage tracks to $Y$ coordinates of the storage track section of
the chart area.

\index{StorageTrackMap!global|(}
The {\tt StorageTrackMap} array contains the usage of storage tracks by
trains presently in storage.  It is indexed by a key formed from the
station name, a plus sign ($+$) character, the track number, a comma
(,), and a pair of times separated by a dash (-). One or the other of
the time pair can be missing.  The values of this array are a pair of
train numbers separated by a greater than sign ($>$).  Again, one or the
other of the train numbers can be missing.
\index{StorageTrackMap!global|)}

\section{Trains array}

\index{Trains!global|(}
The {\tt Trains} array holds information about trains.  It is indexed
by the train number, an integer followed by an optional dash (-) and a
section number, if the train is operated as than one section.  The
value of this array is a list of lists.  The first list is the base
train information as a list containing the name of the train, the speed
of the train in {\em smph}\footnote{Or {\em skph}.}, the train's class
number, and a list of note numbers relating to the train.  The
remainder of lists are the trains station stops as a list of arrival
time or originating track number\footnote{Indicated as the two element
list \{Origin tracknum\}.}, the station, the departure time or arrival
track number\footnote{Indicated as the two element list \{Storage
tracknum\}.}, the cab name or display color, and a list of note numbers
relating to the train at this station.
\index{Trains!global|)}

\section{Time table creation data}

\index{MakeTimeTableStatus!global|(}
The {\tt MakeTimeTableStatus} variable contains the global
configuration information used to generate hard copy time tables.  This
is an array with a number of elements relating to the format
configuration information used to generate hard copy time tables.  The
field names are:

\begin{description}

\item[TimeFormat] The time format to use.  Values can be 24 or 12.

\item[AMPMFormat] The format to use when 12 hour time is selected. 
Values can be a (for small a or p), AP (for large AM or PM), or lB (for
light font for AM, bold font for PM).

\item[Title] The title of the time table, usually the name of the
railroad.

\item[SubTitle] The subtitle of the time table.

\item[Date] The date to be printed.

\item[Filename] The name of the LaTeX file to generate.

\item[TOCP] Flag to indicate if a table of contents should be generated.

\item[NSides] Number of sided printing to generate for, either single or
double. 

\item[GroupBy] The grouping method to group separate times by, Class or
Manually.

\item[StationColWidth] Width, in inches, the station name column should
be.  Default is 1.5.

\item[TimeColWidth] Width, in inches, the time columns should be.
Default is .5.

\end{description}
\index{MakeTimeTableStatus!global|)}



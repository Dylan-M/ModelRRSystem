%* 
%* ------------------------------------------------------------------
%* RaildriverServer.tex - Raildriver Server programming guide
%* Created by Robert Heller on Thu Apr 19 14:22:52 2007
%* ------------------------------------------------------------------
%* Modification History: $Log$
%* Modification History: Revision 1.4  2007/11/30 13:56:50  heller
%* Modification History: Novemeber 30, 2007 lockdown.
%* Modification History:
%* Modification History: Revision 1.3  2007/10/22 17:17:27  heller
%* Modification History: 10222007
%* Modification History:
%* Modification History: Revision 1.2  2007/05/06 19:15:41  heller
%* Modification History: Lock down  for 2.1.8 release candidate 1
%* Modification History:
%* Modification History: Revision 1.1  2002/07/28 14:03:50  heller
%* Modification History: Add it copyright notice headers
%* Modification History:
%* ------------------------------------------------------------------
%* Contents:
%* ------------------------------------------------------------------
%*  
%*     Model RR System, Version 2
%*     Copyright (C) 1994,1995,2002-2005  Robert Heller D/B/A Deepwoods Software
%* 			51 Locke Hill Road
%* 			Wendell, MA 01379-9728
%* 
%*     This program is free software; you can redistribute it and/or modify
%*     it under the terms of the GNU General Public License as published by
%*     the Free Software Foundation; either version 2 of the License, or
%*     (at your option) any later version.
%* 
%*     This program is distributed in the hope that it will be useful,
%*     but WITHOUT ANY WARRANTY; without even the implied warranty of
%*     MERCHANTABILITY or FITNESS FOR A PARTICULAR PURPOSE.  See the
%*     GNU General Public License for more details.
%* 
%*     You should have received a copy of the GNU General Public License
%*     along with this program; if not, write to the Free Software
%*     Foundation, Inc., 675 Mass Ave, Cambridge, MA 02139, USA.
%* 
%*  
%* 

\chapter{Writing clients for the Raildriver Server.}
\label{chapt:rd:RaildriverServer}
\typeout{$Id$}

The PI Engineering Rail Driver Console is a working ``model'' of a
modern locomotive control stand.  It is equipped with a reverser, a
throttle/dynamic brake lever, an automatic (trainline) brake, an
independent (locomotive) brake, plus a collection of additional controls,
buttons, and switches.  This is a USB interface device.  The Model
Railroad System includes a hotplug daemon\index{Rail Driver!hotplug
daemon}\index{USB hotplug daemon|see{Rail Driver, hotplug daemon}},
written using libusb, that interfaces with this device.  The hotplug
subsystem launches this user-mode daemon\index{Rail Driver!user-mode
daemon}, which connects to the Rail Driver Console through the USB
interface and then listens on a TCP/IP port.  Client programs can
connect to the daemon through the TCP/IP port.  Any number of client
programs can connect and each can listen to different events.  If more
than one Rail Driver Console is plugged in, additional daemons are
started and they will listen on different TCP/IP ports. The first
daemon will listen on port 40990, and the second on port 41000 and the
third on port 41010, and so on.

\begin{table}[hbpt]
\begin{centering}
\begin{tabular}{|l|p{3in}|}
\hline
\textbf{Command} & \textbf{Description} \\
\hline
\hline
Exit & Disconnect from the server. \\
\hline
Clear & Clear event mask. \\
\hline
Mask maskbits & Add to the event mask. \\
\hline
Pollvalues maskbits & Get the current event state. \\
\hline
Led digits & Set the speedometer display \\
\hline
Speaker on / off & Turn the speaker on or off. \\
\hline
\end{tabular}
\caption{Rail Driver Client Commands}
\label{tab:rd:rdclient}\index{Rail Driver!Client Commands}
\end{centering}
\end{table}
\begin{table}[hbpt]
\begin{centering}
\begin{tabular}{|l|p{3in}|}
\hline
\textbf{Mask Name} & \textbf{Description} \\
\hline
\hline
Reverser& Reverser lever event.\\
\hline     
Throttle& Throttle / Dynamic brake lever.\\
\hline
Autobrake& Automatic (trainline) brake lever.\\
\hline     
Independbrk& Independent (locomotive) brake lever.\\
\hline     
Bailoff& Independent brake lever bailoff.\\
\hline     
Headlight& Headlight switch.\\
\hline     
Wiper& Wiper switch.\\
\hline     
Digital1& Blue Buttons 1-8.\\
\hline     
Digital2&  Blue Buttons 9-16.\\
\hline     
Digital3& Blue Buttons 17-24. \\
\hline     
Digital4& Blue Buttons 25-28, Zoom up, Zoom down, Pan up, and Pan right.\\
\hline     
Digital5& Pan down, Pan left, Range up, Range down, Emergency brake up,
Emergency brake down, Alert, and Sand. \\
\hline     
Digital6& Pantograph, Bell, Whistle up, and Whistle down. \\
\hline     
 \end{tabular}
\caption{Rail Driver Maskbits Commands}
\label{tab:rd:maskbits}\index{Rail Driver!Maskbits Commands}
\end{centering}
\end{table}
\begin{table}[hbpt]
\begin{centering}  
\begin{tabular}{|l|l|p{2in}|}
\hline
\textbf{Code} & \textbf{Message} & \textbf{Description} \\
\hline
\hline
201 & OK & Generic acknowledgment.\\
\hline
202 & Events: maskbit\verb=[==(bits)\verb=]=& One or more masked events.\\
\hline
299 & GOODBYE & Server is about to close the connection.  Sent in
response to an Exit command.\\
\hline
502 & Parse error& There was an error parsing a command.\\
\hline
503 & message from the parser& A detailed message from the parser.\\
\hline
504 & message: object 'something'& A detailed message from the parser.\\
\hline
\end{tabular}
\caption{Rail Driver Server Messages}
\label{tab:rd:rdserver}\index{Rail Driver!Server Messages}
\end{centering}
\end{table}
The client and server communicate by sending messages as ASCII text
lines.  The defined client messages are shown in
Table~\ref{tab:rd:rdclient}. The maskbits define the events to listen for
or to poll.  They are listed in Table~\ref{tab:rd:maskbits}. The server
sends messages prefixed with a three digit number.  The first digit is a
severity level. A severity level of 2 is informational and a
severity level of 5 is error.  The full set of messages sent by the
server are listed in Table~\ref{tab:rd:rdserver}.

\section{Sample Tcl Client Script}

\lstinputlisting[caption={LocoTest.tcl},label=lst:rd:LocoTest]{LocoTest.tcl}
The Tcl script shown in Listing~\ref{lst:rd:LocoTest} is a simple Tcl program
that connects to the Rail Driver daemon and simulates a locomotive, by
using the reverser, throttle, and independent brake levers to set the
direction, accelerate, and slow down a virtual locomotive and display
its speed on the LED display on the Rail Driver console.

A connection to the Rail Driver daemon is opened on line 136 and the
resulting file channel is set to line buffering on line 138.  Lines
141--179 contain a file event handler for messages coming from the
daemon.  This handler parses the results and updates the global state
variables based on changes in the reverser, throttle, and independent
brake levers.  It also checks for the Alert button and sets a flag if
this button is pressed.  If it looses the connection to the server, it
closes the connection and arranging to exit the program.

Lines 181--228 contain a procedure that is run every 1/2 second.  This
procedure applies updates to the virtual locomotive's
state\footnote{Only its speed, contained in the global variable
\texttt{CurrentSpeed}. To control an actual locomotive, this is where
there would be code (eg using the XPressNet class library) to set the
locomotive's speed.}, taking into account the current settings of the
reverser, throttle, and independent brake levers. It then sends the
current state (speed) to the LED display on the Rail Driver console. It
also checks the Alert flag and arranges to disconnect from the daemon
when the Alert button is pressed.



%* 
%* ------------------------------------------------------------------
%* CMRIProgramming.tex - Using the CMR/I interface.
%* Created by Robert Heller on Thu Apr 19 14:36:42 2007
%* ------------------------------------------------------------------
%* Modification History: $Log$
%* Modification History: Revision 1.3  2007/11/30 13:56:50  heller
%* Modification History: Novemeber 30, 2007 lockdown.
%* Modification History:
%* Modification History: Revision 1.2  2007/10/22 17:17:27  heller
%* Modification History: 10222007
%* Modification History:
%* Modification History: Revision 1.1  2007/05/06 12:49:38  heller
%* Modification History: Lock down  for 2.1.8 release candidate 1
%* Modification History:
%* Modification History: Revision 1.1  2002/07/28 14:03:50  heller
%* Modification History: Add it copyright notice headers
%* Modification History:
%* ------------------------------------------------------------------
%* Contents:
%* ------------------------------------------------------------------
%*  
%*     Model RR System, Version 2
%*     Copyright (C) 1994,1995,2002-2005  Robert Heller D/B/A Deepwoods Software
%* 			51 Locke Hill Road
%* 			Wendell, MA 01379-9728
%* 
%*     This program is free software; you can redistribute it and/or modify
%*     it under the terms of the GNU General Public License as published by
%*     the Free Software Foundation; either version 2 of the License, or
%*     (at your option) any later version.
%* 
%*     This program is distributed in the hope that it will be useful,
%*     but WITHOUT ANY WARRANTY; without even the implied warranty of
%*     MERCHANTABILITY or FITNESS FOR A PARTICULAR PURPOSE.  See the
%*     GNU General Public License for more details.
%* 
%*     You should have received a copy of the GNU General Public License
%*     along with this program; if not, write to the Free Software
%*     Foundation, Inc., 675 Mass Ave, Cambridge, MA 02139, USA.
%* 
%*  
%* 

\chapter{Using the CMR/I (Bruce Chubb) Interface.}
\label{chapt:CMRI:CMRIProgramming}
\typeout{$Id$}

\section{CMri Class}
\begin{lstlisting}[caption={CMri SNIT type (Class), simplified public interface},
		   language=Tcl,label=lst:CMRI:CMriClass]
snit::type CMri {
  option -baud -readonly yes -default 9600 \
		-type {snit::enum -values {9600 19200 28800 57600 115200}}
  option -maxtries -readonly yes -default 10000 \
		-type {snit::integer -min 1000 -max 100000}
  constructor {port args} { }
  destructor { }
  method Inputs {ni {ua 0}} { }
  method Outputs {ports {ua 0}} { }
  method InitBoard {CT ni no ns ua card dl} { }
}

\end{lstlisting}
The CMR/I Interface\cite{internals} has a Tcl API encoded as a Tcl SNIT
type (class) that encapsulates a Bruce Chubb
CMR/I\cite{Chubb89,ChubbBAS04} serial ``bus'', containing one or more
interface nodes, which are instances of SMINI\index{CMR/I!Super Mini
Node} (Super Mini Node), USIC\index{CMR/I!Classic Universal Serial
Interface Card} (Classic Universal Serial Interface Card), or
SUSIC\index{CMR/I!Super Classic Universal Serial Interface Card} (Super
Classic Universal Serial Interface Card). Each card has a unique 6 bit
address and contains a number of input and output ports\footnote{The
SMINI is a self contained card with 3 input ports (24 bits) and 6
output ports (48 bits).}. The SNIT type (class) connects to a serial
port which has an RS232 to RS485 conversion card that in turn connects
to the RS485-based serial bus. This SNIT type (class) exposes three
methods to access cards on the bus, an initialization method to
initialize the board, an input method to read the input ports, and an
output method to write to the output ports. A simplified listing of
this SNIT type (class) is shown in Listing~\ref{lst:CMRI:CMriClass}. 

The constructor takes the name of the port device file, with options
for the BAUD rate and the maximum number of retries. The constructor opens
the serial port and conditions it for unbuffered binary I/O at the
specified BAUD rate. The destructor restores the port and closes it.
The Inputs() method reads the contents of the input ports of a
specified node and the Outputs() method writes to the output ports of a
specified node. The InitBoard() method initializes a specified node.
The InitBoard takes either a packed card type list (for USIC or SUSIC
nodes) or a yellow bi-color LED map (for SMINI nodes), the number of
input\footnote{Must be 3 for SMINI cards.} and output\footnote{Must be
6 for SMINI cards.} ports, the number of yellow bi-color LED
signals\footnote{Only used for SMINI cards.}, the card address, delay
value to use\footnote{Only used for for older USIC cards.}, and the
error message pointer.  

\section{Basic usage}
\begin{lstlisting}[caption={Using the CMR/I from Tcl},
		   label=lst:CMRI:Tcl1]
#
package require Cmri 2.0.0;#		Load the CMR/I snit type.

# The board address of the SMINI card.
global UA
set UA 0


# Connect to the bus on COM3: (/dev/ttyS2), at 9600 BAUD, with
# a retry count of 10000, capturing error messages.
if {[catch {cmri::CMri CMriBus /dev/ttys0 -baud 9600 -maxtries 10000} errorMessage]} {
  # Handle error.
  puts -nonewline stderr "Could not connect to CMR/I bus on /dev/ttyS2: "
  puts stderr "$errorMessage"
  rename CMriBus {}
  exit 99
}

# Initialize SMINI board
if {[catch {CMriBus InitBoard {0 0 0 0 0 0} 3 6 0 $UA SMINI 0} result]} {
  set errorMessage [lindex $result 1]
  # Handle error.
  puts -nonewline stderr "Could not initialize SMINI card at UA "
  puts stderr "$UA: $errorMessage"
  rename CMriBus {}
  exit 99
}

# Initialize port buffer
#	Turn all bits of port 0 on 
#	Turn all of the even bits of port 1 on
#	Turn all of the odd bits of port 2 on
#	Turn the lower nibble of port 3 on
#	Turn the upper nibble of port 4 on
#	Leave port 5 off
  set Outputs {0xFF 0xAA 0x55 0x0F 0xF0 0}
# Write to SMINI ports
if {[catch {CMriBus Outputs $Outputs $UA} result]} {
  # Handle error.
  puts -nonewline stderr "Could not write to the output ports of "
  puts stderr "SMINI card at UA $UA: $result"
  rename CMriBus {}
  exit 99
}

# Read SMINI ports
if {[catch {CMriBus Inputs 3 $UA} result]} {
  puts -nonewline stderr "Could not read from the input ports of "
  puts stderr "SMINI card at UA $UA: $result"
  rename CMriBus {}
  exit 99
}

set Inputs $result

foreach p {0 1 2} v $Inputs {
  puts "Port $p = $v"
}

rename CMriBus {} 
exit 
\end{lstlisting} 
The program shown in Listing~\ref{lst:CMRI:Tcl1} opens a connection to
a CMR/I bus with a SMINI card (Line 10 of Listing~\ref{lst:CMRI:Tcl1}).
It then initialize the SMINI card (Line 20 of
Listing~\ref{lst:CMRI:Tcl1}), write to the output ports of the SMINI
card (Line 38 of Listing~\ref{lst:CMRI:Tcl1}), and then then read from
the input ports of the SMINI card (Line 48 of
Listing~\ref{lst:CMRI:Tcl1}). Finally, the serial port is closed when
the instance is deleted (Line 62 of Listing~\ref{lst:CMRI:Tcl1}) and
the program exits. While this is valid program, it is not partitularly
useful in a Model Railroad control context.  More typically, the
program would have a loop around the I/O functions, repeatedly reading
the inputs, which would typically contain information about the state
of CTC panel controls, track switches (turnouts), and block occupancy
detectors, and code to set the outputs, which would control turnouts
(via switch machines), track side signals, and CTC panel
lamps\footnote{See Chapter~\ref{chapt:CTCPanels} for information about
programming CTC Panels on a computer screen using Tcl/Tk.}.  A more
realistic program is shown in skeleton form in
Listing~\ref{lst:CMRI:Tcl2}.

\section{Advanced usage}
\begin{lstlisting}[caption={Using the CMR/I from Tcl, more realistic version},
		   label=lst:CMRI:Tcl2]
#
package require Cmri 2.0.0;#		Load the CMR/I shared library.

# The board address of the SMINI card.
global UA
set UA 0


# Connect to the bus on COM3: (/dev/ttyS2), at 9600 BAUD, with
# a retry count of 10000, capturing error messages.
if {[catch {cmri::CMri CMriBus /dev/ttys0 -baud 9600 -maxtries 10000} errorMessage]} {
  # Handle error.
  puts -nonewline stderr "Could not connect to CMR/I bus on /dev/ttyS2: "
  puts stderr "$errorMessage"
  rename CMriBus {}
  exit 99
}

# Initialize SMINI board
if {[catch {CMriBus InitBoard {0 0 0 0 0 0} 3 6 0 $UA SMINI 0} result]} {
  set errorMessage [lindex $result 1]
  # Handle error.
  puts -nonewline stderr "Could not initialize SMINI card at UA "
  puts stderr "$UA: $errorMessage"
  rename CMriBus {}
  exit 99
}

while {1} {
  # Read SMINI ports
  if {[catch {CMriBus Inputs 3 $UA} errorMessage]} {
    puts -nonewline stderr "Could not read from the input ports of "
    puts stderr "SMINI card at UA $UA: $errorMessage"
  } else {
    set Inputs $result
    # Process Input ports, generating values for the output ports
    # (Process code goes here)
    # Write to SMINI ports
    if {[catch {CMriBus Outputs $Outputs $UA} errorMessage]} {
      # Handle error.
      puts -nonewline stderr "Could not write to the output ports of "
      puts stderr "SMINI card at UA $UA: $errorMessage"
    }
  }
  after 1000;# Sleep for one second
}
rename CMriBus {} 
exit 
\end{lstlisting} 


\section{Translations from The C/MRI User's Manual V3.0 Examples}

I have translated selected example programs from The C/MRI User's Manual
V3.0\cite{Chubb03}.  The source code files are included in the
documentation directory.

\subsection{From Chapter 9: SMINI Application Examples}
\lstinputlisting[caption={Figure 9-5, Tcl Translation},
		 firstline=37]{fig9-5.tcl}

\subsection{From Chapter 12: SUSIC/USIC Application Examples}
\lstinputlisting[caption={Figure 12-13, Tcl Translation},
		 firstline=37]{fig12-13.tcl}

\subsection{From Chapter 14: Distributed Application Examples}
\lstinputlisting[caption={Figure 14-5, Tcl Translation},
		 firstline=37]{fig14-5.tcl}


%* 
%* ------------------------------------------------------------------
%* AN02_Introduction.tex - Application Note 02: Introduction
%* Created by Robert Heller on Thu Sep 20 08:34:53 2012
%* ------------------------------------------------------------------
%* Modification History: $Log$
%* Modification History: Revision 1.1  2002/07/28 14:03:50  heller
%* Modification History: Add it copyright notice headers
%* Modification History:
%* ------------------------------------------------------------------
%* Contents:
%* ------------------------------------------------------------------
%*  
%*     Model RR System, Version 2
%*     Copyright (C) 1994,1995,2002-2012  Robert Heller D/B/A Deepwoods Software
%* 			51 Locke Hill Road
%* 			Wendell, MA 01379-9728
%* 
%*     This program is free software; you can redistribute it and/or modify
%*     it under the terms of the GNU General Public License as published by
%*     the Free Software Foundation; either version 2 of the License, or
%*     (at your option) any later version.
%* 
%*     This program is distributed in the hope that it will be useful,
%*     but WITHOUT ANY WARRANTY; without even the implied warranty of
%*     MERCHANTABILITY or FITNESS FOR A PARTICULAR PURPOSE.  See the
%*     GNU General Public License for more details.
%* 
%*     You should have received a copy of the GNU General Public License
%*     along with this program; if not, write to the Free Software
%*     Foundation, Inc., 675 Mass Ave, Cambridge, MA 02139, USA.
%* 
%*  
%* 

\chapter{Introduction}
\label{chapt:Introduction}
\typeout{$Id$}

This application note presents the software for a 2' by 6' N scale
continious running oval layout.  This module is a table top module and
features a single track oval with a pair of passing sidings. It is set
up to run two trains in \textit{opposite} directions and does so in a
fully automated way using computer control, using USB connected IR
sensors and control modules from Azatrax.  The layout illustrates how
to automate signaling and control of trains moving in opposing
directions through a single track main line.  The ideas presented in
this Application Note could be adapted for more realistic layouts and
can be incorporated into a CTC system, with automated diversions into
passing sidings.  This layout, as is, could also be used as a
unattended educational museum layout, demostrating how bi-directional
traffic is handled on a single track mainline.

This application note expands on information contained in the
Programming Guides\cite{progguide}, User Manual\cite{userman}, and
Internals Manual\cite{internals}. The \textit{Dispatcher} program,
described in the User Manual, was used to create and maintain the
dispatcher code described in Chapter~\ref{chapt:AutoDispatcher}. This
code makes use of the Tcl API for the Azatrax MRD2-U and SR4 devices,
which is described in the Internals Manual and Programming Guides. The
layout uses some circuits described by Rob
Paisley\cite{MRRElectronicsRPAISLEY}. 


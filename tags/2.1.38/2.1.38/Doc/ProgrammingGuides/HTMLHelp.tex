\chapter{HTMLHelp}

\section{About}

The HTMLHelp widget is a help dialog that display program documentation
coded in a simple subset of HTML.  There is limited support for
Cascading Style Sheets.  There is a support for searching and there is a
simple history stack.

\section{Using the HTMLHelp widget}

The HTMLHelp is a SNIT widgetadaptor based on the BWidget Dialog widget.
It takes these options:

\begin{description}
\item[-textwidth] The width of the help text component (see -width of
the text widget).
\item[-width] The width of the HTMLHelp (see -width of the Dialog
widget).
\item[-height] The height of the HTMLHelp (see -height of the Dialog
widget).
\item[-side] The side to place the pane slider button.  Can be top (the
default) or bottom (see -side of the PanedWindow widget). This is a read
only option.
\item[-helpdirectory] The name of the directory where the HTML code
lives. This is a read only option.  There is no default value and this
option must be specified. (But see below for the typemethod
setDefaults.)
\item[-tableofcontents] The name of the HTML file in the help directory
that contains the Table Of Contents.  This is a read only option. 
There is no default value and this option must be specified. (But see
below for the typemethod setDefaults.)
\end{description}

There is one public method available:

\begin{verbatim}
$HTMLHelpObject helptopic topicstring
\end{verbatim}

This method searches the table of contents for a hyperlink with the
specified help topic string as the link text (case folded search) and
opens the HTMLHelp dialog with the page specified by the link
associated with this text.

There are two public typemethods available:

\begin{verbatim}
HTMLHelp::HTMLHelp setDefaults helpdir tocfile
\end{verbatim}

This method sets the default values for the \verb=-helpdirectory= and
\verb=-tableofcontents= options.

\begin{verbatim}
HTMLHelp::HTMLHelp help topicstring
\end{verbatim}

This typemethod just calls the \verb=helptopic= method for a default help
dialog object.  If a default help does not exist, it is created, using
the default values set by the \verb=setDefaults= typemethod.

\section{Creating Help text with \LaTeX\  and tex4ht}

Help text HTML files can be generated using \LaTeX\  and tex4ht.  You
need be sure you include a table of contents (\verb=\tableofcontents=)
and use the htlatex command with \verb=html,4,info= as its second
parameter. The former creates the index for the left column and the
latter breaks the file up into one file for each section.  Be sure to
include a chapter or section titled ``Help''\footnote{A sample chapter
is included with the development code as Help.tex.}, since this is used
as the topic text for the HTMLHelp dialog itself.  Generally you would
write a chapter, section or subsection for each help topic you expect
to have for your application -- that is for all of the Help buttons on
dialogs and Help menu items on main windows\footnote{See SampleCode.tex
for a sample documentation \LaTeX\  source file.}.





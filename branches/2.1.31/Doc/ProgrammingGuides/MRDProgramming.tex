%* 
%* ------------------------------------------------------------------
%* MRDProgramming.tex - Using the MRD interface
%* Created by Robert Heller on Fri Dec  2 09:35:04 2011
%* ------------------------------------------------------------------
%* Modification History: $Log$
%* Modification History: Revision 1.1  2002/07/28 14:03:50  heller
%* Modification History: Add it copyright notice headers
%* Modification History:
%* ------------------------------------------------------------------
%* Contents:
%* ------------------------------------------------------------------
%*  
%*     Model RR System, Version 2
%*     Copyright (C) 1994,1995,2002-2005  Robert Heller D/B/A Deepwoods Software
%* 			51 Locke Hill Road
%* 			Wendell, MA 01379-9728
%* 
%*     This program is free software; you can redistribute it and/or modify
%*     it under the terms of the GNU General Public License as published by
%*     the Free Software Foundation; either version 2 of the License, or
%*     (at your option) any later version.
%* 
%*     This program is distributed in the hope that it will be useful,
%*     but WITHOUT ANY WARRANTY; without even the implied warranty of
%*     MERCHANTABILITY or FITNESS FOR A PARTICULAR PURPOSE.  See the
%*     GNU General Public License for more details.
%* 
%*     You should have received a copy of the GNU General Public License
%*     along with this program; if not, write to the Free Software
%*     Foundation, Inc., 675 Mass Ave, Cambridge, MA 02139, USA.
%* 
%*  
%* 


\chapter{Using the MRD (USB Model Railroad Detectors from Azatrax) Interface.}
\label{chapt:MRD:MRDProgramming}
\typeout{$Id$}

\section{MRD Class}
\begin{lstlisting}[caption={MRD Class},
		   language=C++,label=lst:MRD:MRDClass]
#define ErrorCode int
class MRD {
public:
	static int NumberOfOpenDevices();
	static char ** AllConnectedDevices();
	enum OperatingMode_Type { NonTurnoutSeparate,
				  NonTurnoutDirectionSensing, 
				  TurnoutSolenoid, 
				  TurnoutMotor };
	MRD(const char *serialnumber, char **outmessage);
	~MRD();
	ErrorCode SetChan1() const;
	ErrorCode SetChan2() const;
	ErrorCode ClearExternallyChanged() const;
	ErrorCode DisableExternal() const;
	ErrorCode EnableExternal() const;
	ErrorCode RestoreLEDFunction() const;
	ErrorCode Identify_1() const;
	ErrorCode Identify_2() const;
	ErrorCode Identify_1_2() const;
	ErrorCode ResetStopwatch() const;
	ErrorCode GetStateData();
	uint8_t PacketCount() const;
	bool Sense_1() const;
	bool Sense_2() const;
	bool Latch_1() const;
	bool Latch_2() const;
	bool HasRelays() const;
	bool ResetStatus() const;
	bool StopwatchTicking() const;
	bool ExternallyChanged() const;
	bool AllowingExternalChanges() const;
	OperatingMode_Type OperatingMode() const;
	void Stopwatch(uint8_t &fract, uint8_t &seconds, uint8_t &minutes,
			uint8_t &hours) const;
	const char *SerialNumber() const;
};
\end{lstlisting}
The MRD Interface\cite{cxxinternals} has both a C++ and a Tcl API and
this chapter will cover both. Both APIs use the same underlying 
code\footnote{In fact the same shared library.}. The API is based on a
C++ class that encapsulates a MRD2-S or MRD2-U device.  Each device has a
factory set serial number.  The MRD2-S has two releays and the MRD2-U
has no relays.  Both devices have two sensors and a stopwatch. The class
instance is connected to a device using its unique serial number. This
class exposes a collection of methods to send command bytes to the
device, a method to retrieve the device's state data and a collection of
methods to access the device's state data, once it has been retrieved.


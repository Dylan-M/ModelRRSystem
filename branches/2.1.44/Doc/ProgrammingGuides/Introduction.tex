%* 
%* ------------------------------------------------------------------
%* Introduction.tex - Introduction
%* Created by Robert Heller on Thu Apr 19 14:24:42 2007
%* ------------------------------------------------------------------
%* Modification History: $Log$
%* Modification History: Revision 1.2  2007/10/22 17:17:27  heller
%* Modification History: 10222007
%* Modification History:
%* Modification History: Revision 1.1  2007/05/06 12:49:38  heller
%* Modification History: Lock down  for 2.1.8 release candidate 1
%* Modification History:
%* Modification History: Revision 1.1  2002/07/28 14:03:50  heller
%* Modification History: Add it copyright notice headers
%* Modification History:
%* ------------------------------------------------------------------
%* Contents:
%* ------------------------------------------------------------------
%*  
%*     Model RR System, Version 2
%*     Copyright (C) 1994,1995,2002-2005  Robert Heller D/B/A Deepwoods Software
%* 			51 Locke Hill Road
%* 			Wendell, MA 01379-9728
%* 
%*     This program is free software; you can redistribute it and/or modify
%*     it under the terms of the GNU General Public License as published by
%*     the Free Software Foundation; either version 2 of the License, or
%*     (at your option) any later version.
%* 
%*     This program is distributed in the hope that it will be useful,
%*     but WITHOUT ANY WARRANTY; without even the implied warranty of
%*     MERCHANTABILITY or FITNESS FOR A PARTICULAR PURPOSE.  See the
%*     GNU General Public License for more details.
%* 
%*     You should have received a copy of the GNU General Public License
%*     along with this program; if not, write to the Free Software
%*     Foundation, Inc., 675 Mass Ave, Cambridge, MA 02139, USA.
%* 
%*  
%* 

\chapter{Introduction}
\label{chapt:Introduction}
\typeout{$Id$}

This manual presents some information about how to write programs that
use various parts of Model Railroad System to control and manage
various aspects of operating your model railroad.  The Model Railroad
System includes code that interfaces with special hardware, including
the PI Engineering Raildriver Console (see
Chapter~\ref{chapt:rd:RaildriverServer}), a Bruce Chubb network of
control nodes (see  Chapter~\ref{chapt:CMRI:CMRIProgramming}), and a
Lenz XPressNet network (see
Chapter~\ref{chapt:XPN:XPressNetProgramming}).  Also included are a
collection of Tcl scripts that implement a collection of widgets that
are useful for various aspects of programming utilities for a model
railroad system.


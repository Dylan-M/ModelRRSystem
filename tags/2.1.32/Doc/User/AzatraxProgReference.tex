%* 
%* ------------------------------------------------------------------
%* MRDTestReference.tex - MRD Test Programs Reference Manual
%* Created by Robert Heller on Thu Nov 10 13:47:22 2011
%* ------------------------------------------------------------------
%* Modification History: $Log$
%* Modification History: Revision 1.1  2002/07/28 14:03:50  heller
%* Modification History: Add it copyright notice headers
%* Modification History:
%* ------------------------------------------------------------------
%* Contents:
%* ------------------------------------------------------------------
%*  
%*     Model RR System, Version 2
%*     Copyright (C) 1994,1995,2002-2005  Robert Heller D/B/A Deepwoods Software
%* 			51 Locke Hill Road
%* 			Wendell, MA 01379-9728
%* 
%*     This program is free software; you can redistribute it and/or modify
%*     it under the terms of the GNU General Public License as published by
%*     the Free Software Foundation; either version 2 of the License, or
%*     (at your option) any later version.
%* 
%*     This program is distributed in the hope that it will be useful,
%*     but WITHOUT ANY WARRANTY; without even the implied warranty of
%*     MERCHANTABILITY or FITNESS FOR A PARTICULAR PURPOSE.  See the
%*     GNU General Public License for more details.
%* 
%*     You should have received a copy of the GNU General Public License
%*     along with this program; if not, write to the Free Software
%*     Foundation, Inc., 675 Mass Ave, Cambridge, MA 02139, USA.
%* 
%*  
%* 

\chapter{MRD Test Programs Reference}
\label{chpt:mrdtest:Reference}
\typeout{$Id$}

These programs can be used to test MRD2-S and MRD2-U units made by
Azatrax. These are infrared sensor units with USB interfaces.  The
MRD2-S include relays for operating switch motors, power relays, or
signals. The MRD2-U contain just a pair of detectors.

\section{MRD Test}

This program is the basic test program and can be used to test basic
functionality of either a MRD2-S or MRD2-U unit.  There are buttons for
each of the commands that can be sent, plus a display area showing the
current state data for the unit.

\subsection{Synopsis}

\begin{verbatim}
MRDTest [X11 Resource Options]
\end{verbatim}

This program takes no parameters.

\section{MRD Sensor Loop}

This program loops, reading the unit sense data at 500 millisecond
intervals, displaying the state of the Sense and Latch bits, plus
whether or not the stopwatch is ticking and the current stopwatch time
value. 

\subsection{Synopsis}

\begin{verbatim}
MRDSensorLoop [X11 Resource Options] sensorSerialNumber
\end{verbatim}

This program takes one parameter, the serial number of the MRD2-S or
MRD2-U unit to test.  The program runs until exited or until the MRD2-S
MRD2-U unit is unplugged.


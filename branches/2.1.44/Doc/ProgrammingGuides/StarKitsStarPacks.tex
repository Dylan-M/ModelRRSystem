%* 
%* ------------------------------------------------------------------
%* StarKitsStarPacks.tex - Creating StarKits and StarPacks
%* Created by Robert Heller on Thu Apr 19 14:46:59 2007
%* ------------------------------------------------------------------
%* Modification History: $Log$
%* Modification History: Revision 1.1  2007/05/06 12:49:39  heller
%* Modification History: Lock down  for 2.1.8 release candidate 1
%* Modification History:
%* Modification History: Revision 1.1  2002/07/28 14:03:50  heller
%* Modification History: Add it copyright notice headers
%* Modification History:
%* ------------------------------------------------------------------
%* Contents:
%* ------------------------------------------------------------------
%*  
%*     Model RR System, Version 2
%*     Copyright (C) 1994,1995,2002-2005  Robert Heller D/B/A Deepwoods Software
%* 			51 Locke Hill Road
%* 			Wendell, MA 01379-9728
%* 
%*     This program is free software; you can redistribute it and/or modify
%*     it under the terms of the GNU General Public License as published by
%*     the Free Software Foundation; either version 2 of the License, or
%*     (at your option) any later version.
%* 
%*     This program is distributed in the hope that it will be useful,
%*     but WITHOUT ANY WARRANTY; without even the implied warranty of
%*     MERCHANTABILITY or FITNESS FOR A PARTICULAR PURPOSE.  See the
%*     GNU General Public License for more details.
%* 
%*     You should have received a copy of the GNU General Public License
%*     along with this program; if not, write to the Free Software
%*     Foundation, Inc., 675 Mass Ave, Cambridge, MA 02139, USA.
%* 
%*  
%* 

\chapter{Creating StarKits and StarPacks}
\label{chapt:StarKitsStarPacks}
\typeout{$Id$}


\lstinputlisting[caption={Makefile fragment for building a StarPack},
		 label={lst:STR:Makefile},
		 firstline=152,lastline=174]{Makefile.am}
The Makefile fragment shown in Listing~\ref{lst:STR:Makefile} builds a
StarPack.  It starts by removing any existing kit or kit
directory\footnote{Basicly, cleaning up from a previously failed
build.}. Then it uses the SDX qwrap and unwrap commands to create a base
level kit directory.  It then uses helper kits to populate this
directory with additional scripts and data files, then finally wraps the
resulting directory into a StarPack with the SDX wrap command (the
\lstinline=-runtime= option makes this a StarPack rather than a
StarKit).

\section{Helper programs included with the Model Railroad System}

Three tclkits are included with the Model Railroad System to help you
build your own programs using elements of the Model Railroad System
Tcl/Tk library of packages:
\begin{description}
\item[AddKitDir.kit] This kit adds a symbolic link to a directory in a
kit directory tree.  This is a ``lazy'' copy of a directory that is usually a
directory containing a Tcl/Tk package.
\item[AddKitFile.kit] This kit adds one or more files to a directory in
a kit directory tree.
\item[MakePkgIndex.kit] This kit runs the Tcl command
\lstinline=pkg_mkIndex= on a directory under the lib directory in a kit
directory tree.
\end{description}
These tclkits help with some of the common tasks involved in build
StarKits and StarPacks.

\subsection{AddKitDir.kit -- Add a directory to a StarKit or StarPack}

This kit does a ``lazy'' copy\footnote{Simply a symbolic link.} of a
directory to a StarKit's or StarPack's vfs\footnote{Virtual File System.}
directory.  It takes three parameters: the name of the kit (without the
\lstinline=.kit= or \lstinline=.vfs= extension), the relative path to
the parent directory, and the path to the directory to add.  A symbolic link,
using the tail of the directory path to add, is made in the StarKit's or
StarPack's \lstinline=.vfs= directory under the relative path specified.

\begin{lstlisting}[label={lst:STR:addkitdir},caption={Adding a directory
to a kit.}]
tclkit AddKitDir.kit MyProgram lib /usr/share/bwidget1.8.0
\end{lstlisting}

The example in Listing~\ref{lst:STR:addkitdir} adds a symbolic link
named \lstinline=bwidget1.8.0= to \lstinline=/usr/share/bwidget1.8.0= in
\lstinline=MyProgram.vfs/lib/=. When the kit is wrapped, the sdx program
will make a deep copy of this directory into the resultant kit or pack.

\subsection{AddKitFile.kit -- Add files to a StarKit or StarPack}

This kit copies one or more files to a directory in
a StarKit's or StarPack's vfs directory. It takes three or more
parameters: the name of the kit (without the \lstinline=.kit= or
\lstinline=.vfs= extension), the relative path to directory to add files
to\footnote{The directory is created if it does not already exist.}, and
one or more files to copy.

\begin{lstlisting}[label={lst:STR:addkitfile},caption={Adding files to a
 kit.}]
tclkit AddKitFile.kit MyProgram lib/packages mypackage1.tcl \
			mypackage2.tcl myextension.so 
\end{lstlisting}

The example in Listing~\ref{lst:STR:addkitfile} adds the files
\lstinline=mypackage1.tcl=, \lstinline=mypackage2.tcl=, and
\lstinline=myextension.so= to the directory
\lstinline=MyProgram.vfs/lib/packages=. 

\subsection{MakePkgIndex.kit -- Create a pkgIndex.tcl file for a
directory in a StarKit or StarPack}

This kit runs \lstinline=pkg_mkIndex= on a directory under the
\lstinline=lib= directory in a StarKit's or StarPack's vfs directory. It
takes two parameters: the name of the kit (without the \lstinline=.kit=
or \lstinline=.vfs= extension) and the directory under the
\lstinline=lib= directory to run \lstinline=pkg_mkIndex= over.

\begin{lstlisting}[label={lst:STR:makepkgindex},caption={Creating the
pkgIndex.tcl for a directory of packages in a kit.}]
tclkit MakePkgIndex.kit MyProgram packages
\end{lstlisting}

The example in Listing~\ref{lst:STR:makepkgindex} runs 
\lstinline=pkg_mkIndex= over the directory
\lstinline=MyProgram.vfs/lib/packages=. 


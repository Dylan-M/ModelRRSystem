%* 
%* ------------------------------------------------------------------
%* AN01_Introduction.tex - AN 01 Introduction
%* Created by Robert Heller on Sun Sep 16 09:54:00 2012
%* ------------------------------------------------------------------
%* Modification History: $Log$
%* Modification History: Revision 1.1  2002/07/28 14:03:50  heller
%* Modification History: Add it copyright notice headers
%* Modification History:
%* ------------------------------------------------------------------
%* Contents:
%* ------------------------------------------------------------------
%*  
%*     Model RR System, Version 2
%*     Copyright (C) 1994,1995,2002-2012  Robert Heller D/B/A Deepwoods Software
%* 			51 Locke Hill Road
%* 			Wendell, MA 01379-9728
%* 
%*     This program is free software; you can redistribute it and/or modify
%*     it under the terms of the GNU General Public License as published by
%*     the Free Software Foundation; either version 2 of the License, or
%*     (at your option) any later version.
%* 
%*     This program is distributed in the hope that it will be useful,
%*     but WITHOUT ANY WARRANTY; without even the implied warranty of
%*     MERCHANTABILITY or FITNESS FOR A PARTICULAR PURPOSE.  See the
%*     GNU General Public License for more details.
%* 
%*     You should have received a copy of the GNU General Public License
%*     along with this program; if not, write to the Free Software
%*     Foundation, Inc., 675 Mass Ave, Cambridge, MA 02139, USA.
%* 
%*  
%* 

\chapter{Introduction}
\label{chapt:Introduction}
\typeout{$Id$}

This application note presents the hardware and software for a 2' by 8'
H0 switching layout.  This module is a table top module and features a
central switch ladder with a number of dead end yard sidings.  Also
included is a ``main line'' track with a push-pull commuter train with
a signaled interlocking plant where this main line goes through the
central switch ladder.  The layout will use an assortment of Azatrax
devices to control turnouts, magnetic uncoupling ramps, and signals. It
will also use Azatrax IR sensors to detect trains at critical places.
It will be using a Lenz DCC system (XPressNet) to operate trains and
will interface to a PI Engineering Rail Driver and use the Rail Driver
as a throttle (for a switcher engine) and for turnout and magnetic
uncoupling ramp control. All of these devices will be connected via USB
to a computer which will manage all operations of the layout using
software elements provided by the Model Railroad System. While the
specific layout presented in this application note is somewhat
contrived, many of the circuits and code fragments would be
applicatical to a more typical model railroad layout.  

